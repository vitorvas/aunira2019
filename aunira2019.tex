\documentclass[svgnames,smaller,table]{beamer}
\usepackage{multirow}
\usepackage{tikz}
\usefonttheme[onlymath]{serif}

\usepackage{listings}
% Configura o listings
\lstset{
  %  basicstyle=\footnotesize,\small,...\tiny
  basicstyle=\ttfamily\scriptsize,
  commentstyle=\color{mygreen},
  numbers=left,
  stepnumber=1,
  showstringspaces=false,
  tabsize=2,
  breaklines=true,
  breakatwhitespace=false
 columns=fixed,
 fontadjust=true,
 basewidth=0.5em
}


\usetheme{lthn}
\setbeamercolor*{normal text}{fg=black}
% -----------------------------------------------------------------------------------------------------------------

\title[Slide]{AUNIRA 2019 e perspectivas para imageamento por nêutrons no CDTN}
\author{Vitor Vasconcelos A. Silva}
\date{\today}
\institute{%
  LTHN - Laboratório de Termo-hidráulica e Neutrônica
  \par
  Serviço de Tecnologia de Reatores - CDTN/CNEN}

\begin{document}

%-------------------------------------------------
\begin{frame}
\titlepage
\end{frame}

%-------------------------------------------------
\begin{frame}
  \frametitle{Summary}
  \tableofcontents%[pausesections]
\end{frame}


\section{CDTN}
%-------------------------------------------------
\begin{frame}
  \frametitle{CDTN}
  \framesubtitle{Nuclear Technology Development Center}
  \begin{center}
    \includegraphics[scale=1.1]{figuras/portaria1_CDTN.jpg}
    \end{center}
\end{frame}

%-------------------------------------------------
\begin{frame}
  \frametitle{CDTN}
  \framesubtitle{Nuclear Technology Development Center}
    \begin{center}
      \includegraphics[scale=1.2]{figuras/predio_28_noite.jpg}
    \end{center}
\end{frame}

%-------------------------------------------------
\begin{frame}
  \frametitle{CDTN}
  \framesubtitle{Nuclear Technology Development Center}
  \begin{itemize}
    \item \textbf{(For now)} Part of Brazilian Nuclear Energy Commission.
  \item Founded in 1952 as IPR (Radioactive Research Institute), part of
    Minas Gerais Federal University (UFMG).
  \item In 1960, TRIGA Mark 1 reactor inaugurated - first criticality.
  \item Many areas related (or not) to nuclear sciences:
    \begin{itemize}
    \item Nuclear waste management;
    \item Environment applications;
    \item Materials science (nanomaterials, graphene applications...);
    \item Radiobiology and radioisotopes production for health applications;
    \item \textbf{Nuclear engineering and Technology;}
    \end{itemize}
  \item Post-graduation program (around 100 students).
%  \item Radioprotection services.
  \end{itemize}
  \begin{center}
    \url{http://www.cdtn.br/en/}
    \end{center}
\end{frame}


\section{Thermal-Hydraulics and Neutronics Laboratory - LTHN}
%-------------------------------------------------
\begin{frame}
  \frametitle{Thermal-Hydraulics and Neutronics Laboratory - LTHN}
  \framesubtitle{Experimental facilities}
  \begin{center}
    \includegraphics[scale=0.2]{figuras/labth.jpg}
  \end{center}
\end{frame}

\begin{frame}
  \frametitle{Thermal-Hydraulics and Neutronics Laboratory - LTHN}
  \framesubtitle{Computer Laboratory}
  \begin{center}
    \includegraphics[scale=0.3]{figuras/lthn-computers-cropped.jpg}
  \end{center}
\end{frame}

%-------------------------------------------------
\begin{frame}
  \frametitle{Thermal-Hydraulics and Neutronics Laboratory - LTHN}
  \framesubtitle{Main activities}
  \begin{center}
    \begin{itemize}
    \item Experimental thermal-hydraulics: spacer grids, counter current fluid flow, PIV.
    \item IPR-R1 TRIGA modelling, uncertainty propagation in MC simulations.
    \item Fusion-fission simulations, advanced fuel burn-up (\textbf{SERPENT}).
    \item Software development ($\star$).
    \item Modelling and simulations of RMB.\\ (Brazilian Multipurpose Reactor $\rightarrow$ detailed project phase).
    \end{itemize}
  \end{center}
\end{frame}

\section{Monte Carlo simulation work}
%-------------------------------------------------
\begin{frame}
  \frametitle{Serpent2}
%  \framesubtitle{Main projects exclusivelly using Serpent2}
  \textbf{Projects exclusively using Serpent2}
  \vspace{10px}
  \begin{enumerate}
    \item OpenFOAM + Serpent2 coupling;
    \item Hybrid fusion-fission system;
    \item ADS simulations with Thorium and Uranium;
    \end{enumerate}
%    \vspace{10px}
%  \textbf{Algo aqui?}
%    \begin{itemize}
%    \item bla
%    \end{itemize}
\end{frame}

\subsection{OpenFOAM + Serpent2 coupling}
%-------------------------------------------------
\begin{frame}
  \frametitle{OpenFOAM + Serpent2 coupling}
  \framesubtitle{Problem description}
  \begin{center}
    \alert{Fidelity on the simulation of nuclear systems}\\
    \vspace{10px}
    \begin{itemize}
    \item RMB: Brazilian Multipurpose Reactor $\rightarrow$ under design.
    \item Why CFD + Monte Carlo $\rightarrow$ High accuracy level.
    \item Initially a simplified model $\rightarrow$ fuel pin.
    \end{itemize}
    \vspace{10px}
    \alert{But why modelling a fuel pin for a plate fuel reactor?}
    \vspace{10px}
    \begin{itemize}
    \item Previous work \cite{Vasconcelos2018}, a fine mesh for a \textbf{TRIGA} fuel/pin;
    \item Straightforward to extend the methodology to an approach using MC (Serpent \texttt{ifc} interface);
    \end{itemize}
  \end{center}
\end{frame}

\begin{frame}
  \frametitle{OpenFOAM + Serpent2 coupling}
  \framesubtitle{Preliminary Results}
  \begin{center}
    \includegraphics[scale=0.55]{figuras/power.png}
    
  \end{center}
\end{frame}

\begin{frame}
  \frametitle{OpenFOAM + Serpent2 coupling}
  \framesubtitle{Preliminary Results}
  \begin{center}
    \includegraphics[scale=0.14]{figuras/reactionrate.png}
    
  \end{center}
\end{frame}

\begin{frame}[fragile] % Coloca isso no frame que tem verbatim
  \frametitle{OpenFOAM + Serpent2 coupling}
  \framesubtitle{Current issues}
  \begin{center}
  Zirconium hydride: the pin modelled is actually a TRIGA fuel element.\\
\begin{verbatim}
  Fatal error in function OTFSabScattering:
  Energy grids differ in OTF S(a,b) interpolation
  Simulation aborted.
\end{verbatim}

On file \texttt{otfsabscattering.c} we get:

\begin{verbatim}
/* NOTE: tää on kohtuullisen harvinainen sirontalaki, johon */
/* törmää esim. h/zr ja zr/h -kirjastoissa. Ei ole kunnolla */
/* testattu. */
\end{verbatim}
  
  \end{center}
\end{frame}

\begin{frame}[fragile] % Coloca isso no frame que tem verbatim
  \frametitle{OpenFOAM + Serpent2 coupling}
  \framesubtitle{Current issues}  
  'Same nuclide on different materials when using \texttt{ifc 9} interface gets only the lower value
  of temperature read. For example: a system with water at 310K and fuel at 423K, where water and fuel
  contains hydrogen. (\texttt{1001.03c})'.
  \vspace{10px}
  'This issue gives a non-negligible differences in $K_{eff}$ and reaction rates (tested). In this case,
  apparently, TMS was not used.'
\end{frame}

% END SUBSECTION ----------------------------------


% BEGIN SUBSECTION --------------------------------------
\subsection{Fusion-fission simulation}
%-------------------------------------------------

\begin{frame}
  \frametitle{Fusion-fission simulation}
  \framesubtitle{Problem description}
  \begin{itemize}
  \item Simulation of a hybrid fusion-fission system.
  \item Geometry: concentric spheres, nine zones filled with RFS with thorium and ten zones with coolant Li$_{17}$Pb$_{83}$.
  \item The source was produced by the D-T fusion reaction generating neutrons of $14.1$ MeV and placed in the central sphere with a radius of 250 cm.
  \end{itemize}
\end{frame}

%-------------------------------------------------
\begin{frame}
  \frametitle{Fusion-fission simulation}
  \framesubtitle{}
  %We are using SERPENT in external source mode to simulate a hybrid system fusion-fission. The geometry simulated is based on concentrically spheres, where there are nine zones filled with the RFS with thorium and ten zones with coolant Li$_{17}$Pb$_{83}$. The source was produced by the D-T fusion reaction generating neutrons of $14.1$ MeV and placed in the central sphere with a radius of 250 cm, as shown in Figure \ref{fusion}.

  \framesubtitle{View}
  \begin{center}
    \includegraphics[scale=0.4]{figuras/fusion.png}
%    \caption{Fusion system geometry.}
%    \label{fusion}
  \end{center}
\end{frame}



%-------------------------------------------------
\begin{frame}
  \frametitle{Fusion-fission simulation}
  \framesubtitle{Results}
  

\begin{table}[htb!]
\caption{k$_{eff}$ Results for different NPS}
\label{NPS}
\centering
\vspace{0.5cm}
\begin{tabular}{c|c|c}\hline
NPS & k$_{eff}$(analog) & 95\% confidence interval\\ \hline
$10000$ & $0.59490$ & $0.53182-0.65798$\\ \hline
$20000$ & $0.69282$ & $0.64750-0.73802$\\ \hline
$30000$ & $0.74796$ & $0.71240-0.78352$\\ \hline
$40000$ & $0.77101$ & $0.74249-0.79953$\\ \hline
$50000$ & $0.76816$ & $0.73942-0.79690$\\ \hline
$60000$ & $0.79388$ & $0.76734-0.82042$\\ \hline
$100000$ & $0.82596$ & $0.80478-0.84714$\\ \hline
${\bf 500000}$ & $0.88611$ & $0.87775-0.89447$\\ \hline
${\bf 1000000}$ & $0.89095$ & $0.88547-0.89643$\\ \hline
${\bf 10000000}$ & $0.90109$ & $0.89905-0.90313$\\ \hline
${\bf 20000000}$ & $0.90142$ & $0.90012-0.90272$\\ \hline
\end{tabular}
\end{table}
\end{frame}

%-------------------------------------------------
\begin{frame}
  \frametitle{Fusion-fission simulation}
  \framesubtitle{Current issues}

  'We are having problems to understand how NPS value influences the results of k$_{eff}$(analog) in those simulations. It was noticed that the values of k$_{eff}$(analog) increases considerably when the value of NPS is also increased. Even for high values of NPS (larger than 500000), the values of  k$_{eff}$(analog) obtained for various NPS are considerably different. %Table \ref{NPS} shows the k$_{eff}$(analog) values obtained for several NPS values.'
  \vspace{10px}

'Similar behavior is verified when we use SERPENT to simulate ADS. Is there an inferior limit to NPS?'
\end{frame}


\subsection{ADS}
%-------------------------------------------------
\begin{frame}
  \frametitle{ADS}
  \framesubtitle{Problem description}
  \textbf{Four ADS cases}
  \begin{enumerate}
    \item GANEX fuel spiked with 50\% of thorium;
  \item GANEX fuel spiked with 50\% of depleted uranium;
  \item UREX$+$ fuel spiked with 50\% of thorium;
  \item UREX$+$ fuel spiked with 50\% of depleted uranium;
  \end{enumerate}
  \vspace{10px}
  \textbf{Geometry}\\
  the subcritical core is a cylinder of $12.0m^3$ filled with a hexagonal lattice formed by 120 $^{232}ThO_2$ rods (gray fuel rods) and 36 rods with reprocessed fuel (green fuel rods). Lead was used as a coolant and as a reflector.
\end{frame}

%-------------------------------------------------
\begin{frame}
  \frametitle{ADS}
  \framesubtitle{Problem description}
  \textbf{Materials}\\
  \vspace{10px}
  'For all materials, cross-sections libraries available in SERPENT were specified at working temperature, which is 1200 K for containing fissile/fissionable material and 900 K for the remaining regions. The parameters for the simulated particle population in external source mode were set to run 2 million source neutrons. The burn-up calculation was performed for 10 years with the same parameters in both cases and all nuclides were included in the \texttt{dep.m} output file.'  

\end{frame}

%-------------------------------------------------
\begin{frame}
  \frametitle{ADS}
  \framesubtitle{View}
  \begin{center}
    \includegraphics[scale=0.1]{figuras/50Th_geom1.png}
    \end{center}
\end{frame}

%-------------------------------------------------
\begin{frame}
  \frametitle{ADS}
  \framesubtitle{Results}
  \begin{table}%[htb!]
    \caption{Computational time}
    \label{time}
    \centering
    \vspace{0.5cm}
    \begin{tabular}{l|r}\hline   
      Case & Computational time\\ \hline
      Case 1 (GANEX + Th) & $194:18:44 $\\ \hline
      Case 2 (GANEX + U) & $287:25:16 $\\ \hline
      Case 3 (UREX + Th) & $208:41:45 $\\ \hline
      Case 4 (UREX + U) & $300:43:54 $\\ \hline
    \end{tabular}
  \end{table}
  
\end{frame}

%-------------------------------------------------
\begin{frame}
  \frametitle{ADS}
  \framesubtitle{Current issue (?)}
  'When we use reprocessed fuel spiked with uranium the computational time is considerably higher. Any ideas why?'
\end{frame}


\section{Conclusions}
%-------------------------------------------------
\begin{frame}
  \frametitle{Conclusions related to Serpent2}
  \framesubtitle{And some future intentions}
  \begin{itemize}
  \item Roughly, half of the activities of neutronic modelling and simulation at the LTHN are done using Serpent2.
  \item Brings reliable results;
  \item Relatively easy to use - multiphysics interface;
  \item Wiki-page is better than the manual...
  \item ... However the sensation that information is scattered, no easy way to go straight to related information to the topic being read.
  \end{itemize}
\end{frame}



%-------------------------------------------------
% FIM
%-------------------------------------------------
\begin{frame}
 \vfill
  \begin{beamercolorbox}[center]{title}
     \Huge{Thank you!}
  \end{beamercolorbox}
  \vfill
\end{frame}

% ----------------------------
% References
\begin{frame}
    \frametitle{References}
    \bibliographystyle{apalike}
    \bibliography{9thUGM.bib}
\end{frame}
\end{document}

